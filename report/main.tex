\documentclass{article}

% Packages nécessaires
\usepackage[utf8]{inputenc} % Encodage UTF-8
\usepackage[T1]{fontenc}   % Encodage des polices
\usepackage{amsmath, amssymb, amsthm} % Mathématiques
\usepackage{geometry}      % Mise en page
\usepackage{graphicx}      % Inclusion d'images
\usepackage{hyperref}      % Hyperliens
\usepackage{xcolor}        % Couleurs pour les liens

\usepackage[authoryear]{natbib}  % Pour le style auteur, année
\usepackage{cite}


\geometry{a4paper, margin=1in}
\hypersetup{
    colorlinks=true,
    linkcolor=blue,
    filecolor=magenta,
    urlcolor=cyan,
    pdftitle={Rapport de Mathématiques},
    pdfauthor={Votre Nom}
}

\theoremstyle{definition}
\newtheorem{definition}{Définition}[section]
\theoremstyle{plain}
\newtheorem{theorem}{Théorème}[section]
\newtheorem{lemma}[theorem]{Lemme}
\newtheorem{proposition}[theorem]{Proposition}
\newtheorem{corollary}[theorem]{Corollaire}

\theoremstyle{remark}
\newtheorem{remark}{Remarque}[section]

\newcommand{\R}{\mathbb{R}}
\newcommand{\N}{\mathbb{N}}

\newcommand{\includesection}[1]{\input{sections/#1}}


\begin{document}

\title{\textbf{\Large{Projet d'Approfondissement en Finance:\\ Net Zero Investing With Impact}}}
\author{Ahyerre Victoire, Crouzet Florent\\
 Moutet Maxime, Selles Etienne \\
\small ENSAE}
\date{\today}
\maketitle

\begin{abstract}
\end{abstract}


\includesection{introduction}
\includesection{definitions-and-metrics}
\includesection{description-method}
\includesection{conclusion}




\bibliographystyle{plain}  % Style pour les citations auteur, année
\bibliography{bibliography}
\end{document}
