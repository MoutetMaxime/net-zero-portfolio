\section{Description of the method}
We describe here the method used in \citep{barahhou2022netzero} to define and construct a net zero portfolio. We have a dynamic approach, where we want to find the optimal portfolio $x^*(t)$ at each date $t$.
To construct our portfolio, we want to solve the following optimization problem.

\paragraph{Optimization problem}
Let $x$ be a portfolio and $\Sigma$ the covariance matrix of stock returns. We want to minimize the tracking error variance of the portfolio $x$ with respect to a benchmarck $b$, subject to a carbon reduction constraint:

\begin{equation}
    x^*(t) = 
    \text{argmin} \frac{1}{2}(x - b(t))^T \Sigma (t)(x - b(t)), \quad
    \text{s.t.} 
    \begin{cases}
        \mathcal{CI}(t, x) \leq \big(1 - \mathcal{R}(t_0, t) \cdot \mathcal{CI}(t_0, b(t_0))\big) \\
        x \in \Omega_1\cap \Omega_2(t)
    \end{cases}
\end{equation}
Where $\mathcal{CI}(t, x)$ the carbon intensity at time $t$ of the portfolio $x$, and $\mathcal{CI}(t, b(t))$ is the carbon intensity at time $t$ of the benchmark. $\Omega_1$ and $\Omega_2(t)$ are sets of additional constraints. 
