\section{Introduction}
The financial sector plays a crucial role in achieving the temperature goals outlined in the Paris Agreement by reallocating financial resources from polluting industries to more sustainable ones. This process is embodied in the concept of portfolio alignment, which involves building investment portfolios whose emission profiles align with global emissions scenarios consistent with the temperature targets. Portfolio decarbonization can be approached in three main ways:

\begin{itemize}
    \item \textbf{Divestment and reinvestment:} Redirect investments from industries with high greenhouse gas emissions to those with lower emissions. However, this approach faces challenges since certain polluting industries, like steel production, are essential for the transition, and there are currently not enough green industries to absorb these redirected investments.\\
    \item \textbf{Supporting transitional companies:} Invest in companies that are currently high emitters but are on track to significantly reduce their emissions over the medium to long term. This method is more complex as it requires estimating companies' future emission trajectories. Nevertheless, it is more effective, as it ensures funding for essential industries while avoiding a significant reduction in the pool of investable companies.\\
    \item \textbf{Impact investing:} Focus on companies likely to use additional funding to implement projects aimed at reducing emissions, which they might not pursue without external financial support.\\
\end{itemize}


Early methodologies for portfolio decarbonization primarily relied on the first strategy. Recent approaches have incorporated both the first and second strategies. This project, conducted in partnership with Kepler Cheuvreux, an independant European financial services company, aims to explore how elements of impact investing can be integrated into portfolio decarbonization strategies.


Our approach throughout this first part of the year is based on the article \citep{barahhou2022netzero}. The objective of this article is to construct a portfolio which satisfies certain conditions both on the decarbonization and the net zero investing of its assets. 
